% !root= Exam 4.tex
\documentclass{article}
\usepackage{tikz}
\usepackage{pgfplots}
\usepackage{setspace}
\usepackage{units}
\usepackage{graphicx}
\usepackage{amsopn}
\usepackage{bbding}
\usepackage{amsmath}
\usepackage{hyperref}
\usepackage{cancel}
\usepackage{etoolbox}
\usepackage{enumitem}
\usepackage{gensymb}
\usepackage{amssymb}
\usepackage{multicol}
\usepackage{numerica}
\usepackage[top=1in, left=1in, right=1in, bottom=1in]{geometry}
\AtBeginEnvironment{document}{\everymath{\displaystyle}}
\pagestyle{plain}
\title{EXAM 4}
\author{MATH 1B, Bach}
\date{Tejas Patel, Roll \#30}
\begin{document}
\maketitle

    

\section{Identifying Quadric Surfaces}
\subsection*{I}
Horizontal Traces (xy plane): Ellipse \\Vertical Traces (xz plane): Parabola\\Vertical Traces (yz plane): Parabola \\Surface: Elliptic Paraboloid
\subsection*{II}
Horizontal Traces(xy plane): Ellipse \\Vertical Traces (xz plane): Hyperbola\\Vertical Traces (yz plane): Hyperbola \\Surface: Hyperboloid of One Sheet
\subsection*{III}
Horizontal Traces (xy plane): Hyperbola \\Vertical Traces (xz plane): Ellipse\\Vertical Traces (yz plane): Ellipse \\Surface: Elliptic Cone
\subsection*{IV}
Horizontal Traces(xy plane): Hyperbola \\Vertical Traces(xz plane): Hyperbola\\Vertical Traces(yz plane): Ellipse \\Surface: Hyperboloid of Two Sheets
\subsection*{V}
Horizontal Traces(xy plane): Hyperbola \\Vertical Traces(xz plane): Parabola\\Vertical Traces(yz plane): Parabola \\Surface: Hyperbolic Paraboloid
\subsection*{VI}
Horizontal Traces(xy plane): Ellipse \\Vertical Traces(xz plane): Ellipse\\Vertical Traces(yz plane): Ellipse \\Surface: Ellipsoid\\[1in]
\begin{multicols*}{2}
    \subsection{For equation (a): $ \frac{x^2}{36}+\frac{y^2}{9}+\frac{z^2}{16}=1$}
Let $k=z$ (to test the $x/y$ plane shape, the horizontal tracing)
\vfill$ \frac{x^2}{36}+\frac{y^2}{9}=1-\frac{z^2}{16}$ is an equation for an \boxed{$ellipse$}
\vfill$ 1-\frac{z^2}{16}\neq0$
\vfill$ 1 \neq \frac{z^2}{16}$
\vfill$ 16 \neq z^2$
\vfill$ k \epsilon \mathbb{R} \;\;\; z \neq \pm 4$
\vfill Now, let $k=y$ (to test the $x/z$ plane shape, the vertical tracing)
\vfill$ \frac{x^2}{36}+\frac{k^2}{9}+\frac{z^2}{16}=1$
\vfill$ \frac{x^2}{36}+\frac{z^2}{16}=1-\frac{k^2}{9}$ is an equation for an \boxed{$ellipse$}
\vfill$ 1-\frac{k^2}{9} \neq 0$
\vfill$ 1-\frac{k^2}{9} \neq 0$
\vfill$ 1\neq \frac{k^2}{9}$
\vfill$ 9 \neq k^2$
\vfill$k \epsilon \mathbb{R} k \neq \pm 3$
\vfill Now, let $k=x$ (to test the $y/z$ plane trace, the vertical tracing)
\vfill$ \frac{k^2}{36}+\frac{y^2}{9}+\frac{z^2}{16}=1$
\vfill$ \frac{y^2}{9}+\frac{z^2}{16}=1-\frac{k^2}{36}$ is an equation for an \boxed{$ellipse$}
\vfill$ 1-\frac{k^2}{36}\neq 0$
\vfill$ 1 \neq \frac{k^2}{36}$
\vfill$ 36 \neq k^2$
\vfill$ k \epsilon \mathbb{R} \;\;\; k \neq \pm 6$
\vfill As the traces on all 3 planes are ellipses, \boxed{$the shape is an \textbf{ellipsoid, figure VI} (6)$}\\[1in]
\vfill\null
\columnbreak
\subsection*{For equation (b): $ \frac{x^2}{4}+\frac{y^2}{16}-z=0$}
Let $k=z$ (to test the $x/y$ plane shape, the horizontal tracing)
\vfill$ \frac{x^2}{4}+\frac{y^2}{16}-k=0$
\vfill$ \frac{x^2}{4}+\frac{y^2}{16}=k$ This is an equation for an \boxed{$Ellipse$}
\vfill$ k \epsilon \mathbb{R} \;\;\; k \neq 0$
\vfill Now, let $k=x$ (to test the $y/z$ plane shape, the vertical tracing)
\vfill$ \frac{k^2}{4}+\frac{y^2}{16}-z=0$
\vfill$ \frac{y^2}{16}-z=-\frac{k^2}{4}$ This is an equation for a \boxed{$Parabola$}
\vfill$-\frac{k^2}{4} \neq 0$
\vfill${k^2} \neq 0$
\vfill$k \epsilon \mathbb{R} \;\;\;k \neq 0$
\vfill Now, let $k=y$ (to test the $x/z$ plane shape, the vertical tracing) 
\vfill $ \frac{x^2}{4}+\frac{k^2}{16}-z=0$
\vfill$ \frac{x^2}{4}-z=-\frac{k^2}{16}$ This is an equation for a \boxed{$Parabola$}
\vfill$ 0 \neq -\frac{k^2}{16}$
\vfill$ 0 \neq k^2$
\vfill$ k \epsilon \mathbb{R} \;\;\; k \neq 0$
\vfill Since the equation $ \frac{x^2}{4}+\frac{y^2}{16}-z=0$ can be rewritten as $ \frac{x^2}{4}+\frac{y^2}{16}=\frac{z}{1}$the equation represents an \boxed{$ \textbf{Elliptic Paraboloid} where $c=\pm1$, $a=\pm2$, and $b=\pm4$$} and matches with \boxed{\textbf{Figure I}}\\[2in]
\end{multicols*}
\begin{multicols*}{2}
\subsection*{For equation (c): $ \frac{x^2}{4}+\frac{y^2}{16}-\frac{z^2}{4}=1$} 
Let $k=z$ (to test the $x/y$ plane shape, the horizontal tracing)
\vfill$ \frac{x^2}{4}+\frac{y^2}{16}-\frac{k^2}{4}=1$
\vfill$ \frac{x^2}{4}+\frac{y^2}{16}=1+\frac{k^2}{4}$ is an equation for an \boxed{$Ellipse$}
\vfill$ 1+\frac{k^2}{4}\neq 0$
\vfill$ \frac{k^2}{4}\neq -1$ has no solutions in the real number spectrum
\vfill Meaning $k \epsilon \mathbb{R} $
\vfill Now, let $k=x$ (to test the $y/z$ plane shape, the vertical tracing)
\vfill$ \frac{k^2}{4}+\frac{y^2}{16}-\frac{z^2}{4}=1$
\vfill$ \frac{y^2}{16}-\frac{z^2}{4}=1-\frac{k^2}{4}$ is an equation for a \boxed{$Hyperbola$}
\vfill$ 1-\frac{k^2}{4}\neq 0$
\vfill$ 1 \neq \frac{k^2}{4}$
\vfill$ 4 \neq k^2$
\vfill$ k \epsilon \mathbb{R} \;\;\; k \neq \pm 2$
\vfill Now, let $k=y$ (to test the $x/z$ plane shape, the vertical tracing)
\vfill$ \frac{x^2}{4}+\frac{k^2}{16}-\frac{z^2}{4}=1$
\vfill$ \frac{x^2}{4}-\frac{z^2}{4}=1-\frac{k^2}{16}$ is an equation for a \boxed{$Hyperbola$}
\vfill$ 1-\frac{k^2}{16} \neq 0$
\vfill$ 1 \neq \frac{k^2}{16} $
\vfill$ 16 \neq {k^2} $
\vfill$ k \epsilon \mathbb{R} \;\;\; k \neq \pm 4 $
\vfill The equation $ \frac{x^2}{4}+\frac{y^2}{16}-\frac{z^2}{4}=1$ is an equation for a \boxed{$\textbf{Hyperboloid of One Sheet}$} where $a=c=\sqrt{4} = 2$ and $b=\sqrt{16}=4$, matching \boxed{$\textbf{Figure II}$}\\[1in]
\vfill\null
\columnbreak
\subsection*{For equation (d): $ x^2-4y^2-z^2=4$}
Let $k=z$ (to test the $x/y$ plane shape, the horizontal tracing)
\vfill $ x^2-4y^2-k^2=4$
\vfill $ x^2-4y^2=4+k^2$ is an equation of a \boxed{$Hyperbola$}
\vfill $ 4+k^2 \neq 0$ 
\vfill $ k^2 \neq -4$ is true for all real numbers
\vfill$ k \epsilon \mathbb{R}$
\vfill Now, let $k=x$ (to test the $y/z$ plane shape, the vertical tracing)
\vfill $ k^2-4y^2-z^2=4$
\vfill $ -4y^2-z^2=4-k^2$
\vfill $ 4y^2+z^2=k^2-4$ is an equation for an \boxed{$Ellipse$}
\vfill $ k^2-4 \neq 0$
\vfill $ k^2 \neq 4$
\vfill $ k \epsilon \mathbb{R}\;\;\; k \neq \pm 2$
\vfill Now, let $k=y$ (to test the $x/z$ plane shape, the vertical tracing)
\vfill $ x^2-4k^2-z^2=4$
\vfill $ x^2-z^2=4+4k^2$ is an equation of a \boxed{$Hyperbola$}
\vfill $ 4+4k^2 \neq 0$
\vfill $ 4k^2 \neq -4$
\vfill $ k^2 \neq -1$ is true for all real numbers
\vfill$ k \epsilon \mathbb{R}$
\vfill The equation $ x^2-4y^2-z^2=4$ can be rewritten as $ \frac{x^2}{4}-y^2-\frac{z^2}{4}=1$ making the quadric a \boxed{$\textbf{Hyperboloid of Two Sheets}.$} The $x$ and $z$ variables are both negative, making the graph center go through the $x$ axis, corresponding to \boxed{$\textbf{Figure IV}$}\\[1in]
\end{multicols*}
\subsection*{For equation (e): $ 9y^2-4x^2-9z=0$}
Let $k=z$ (to test the $x/y$ plane shape, the horizontal tracing)
\vfill $ 9y^2-4x^2-9k=0$
\vfill $ 9y^2-4x^2=9k$ is an equation of a \boxed{$Hyperbola$}
\vfill $ 9k \neq 0$
\vfill $ k \epsilon \mathbb{R}\;\;\; k \neq 0$
\vfill Now, let $k=x$ (to test the $y/z$ plane shape, the vertical tracing)
\vfill $ 9y^2-4k^2-9z=0$
\vfill $ 9y^2-9z=4k^2$ is an equation of a \boxed{$Parabola$}
\vfill $ 4k^2 \neq 0$
\vfill $ k \epsilon \mathbb{R}\;\;\;  k \neq 0$
\vfill Now, let $k=y$ (to test the $x/z$ plane shape, the vertical tracing)
\vfill $ 9k^2-4x^2-9z=0$
\vfill $ -4x^2-9z=-9k^2$ is an equation of a \boxed{$Parabola$}
\vfill $ -9k^2 \neq 0$
\vfill $ k^2 \neq 0$
\vfill $ k \epsilon \mathbb{R}\;\;\;  k \neq 0$
\vfill The equation $ 9y^2-4x^2-9z=0$ carries the format of a \boxed{$\textbf{Hyperbolic Paraboloid}, represented in \textbf{Figure V}$}, where the z axis value is the non-exponential and the y variable is the positive squared variable\\[3in]

\pagebreak
\section{$ \sum_{n=1}^{\infty}  (-1)^n \frac{(x+5)^n}{n\cdot 3^n}$}
\begin{multicols*}{2}
    

\subsection*{a) Finding the radius of convergence}
Method: Ratio Test $R =  \lim_{n\to\infty}\left| \frac{a_{n+1}}{a_n} \right|$ where the series converges if $|R|<1$
\\[0.1in]and where $a_n$ denotes the $n_{th}$ term of the series and $r$ is the radius of convergence
\\[0.1in] $a_n= (-1)^n \frac{(x+5)^n}{n\cdot 3^n}$
\\[0.1in]$ R=\lim_{n \to \infty}  \left| \dfrac{(-1)^{\cancelto{}{n+1}} \dfrac{(x+5)^{n+1}}{(n+1)\cdot 3^{n+1}}}{\cancelto{1}{(-1)^n} \dfrac{(x+5)^n}{n\cdot 3^n}}\right|$
\\[0.1in]$ R=\lim_{n \to \infty}  \left| \dfrac{(x+5)^{\cancelto{}{n+1}}}{(n+1)\cdot 3^{\cancelto{}{n+1}}}\cdot \dfrac{n\cdot \cancelto{1}{3^n}}{\cancelto{1}{(x+5)^n}}\right|$
\\[0.1in]$ R=\lim_{n \to \infty}  \left| \dfrac{n\cdot(x+5)}{3\cdot(n+1)}\right|$
\\[0.1in]$ R =\left|x+5\right| \lim_{n \to \infty}  \left| \dfrac{n}{3\cdot(n+1)}\right|$
\\[0.1in]$ R =\left|{x+5}\right| \cdot \left|\frac{1}{3}\right|$
\\[0.1in]$ \left|\dfrac{x+5}{3}\right| = 1$
\\[0.1in]$ \left|{x+5}\right| = 3$
\\[0.1in]\boxed{r= 3}
\\[2in]\boxed{\textbf{The radius of convergence is 3}}
\\\boxed{$\textbf{and the interval of convergence is} $-8  < x \leq -2}
\vfill\null
\columnbreak
\subsection*{b) Finding the Interval of Convergence}
The series converges if $ \left|\dfrac{x+5}{3}\right|<1$
\\[0.1in]$-1< \dfrac{x+5}{3}<1$
\\[0.1in]$-3<{x+5}<3$
\\[0.1in]$-8<x<-2$
\\\textbf{Testing the endpoints:}
\\\textbf{Lower:} $ \sum_{n=1}^{\infty}  (-1)^n \frac{(-8+5)^n}{n\cdot 3^n}$
\\[0.1in]$ \sum_{n=1}^{\infty}  (-1)^n \frac{(-3)^n}{n\cdot 3^n}$
\\[0.1in]The $\dfrac{(-3)^n}{3^n}$ can be reduced to $(-1)^n$
\\[0.1in]$ \sum_{n=1}^{\infty}  (-1)^n \frac{(-1)^n}{n}$
\\[0.1in]The ${(-1)^n}{(-1)^n}$ can be reduced to $(-1)^{2n}$, which can be reduced to $1^n$, which is equal to 1 for all real numbers
\\[0.1in]$ \sum_{n=1}^{\infty}\frac{1}{n}$ \boxed{$diverges by the Harmonic Series Test$}, meaning -8 can not be included in the interval of convergence
\\[0.1in]\textbf{Upper:} $ \sum_{n=1}^{\infty}  (-1)^n \frac{(-2+5)^n}{n\cdot 3^n}$
\\[0.1in]$ \sum_{n=1}^{\infty}  (-1)^n \frac{3^n}{n\cdot 3^n}$
\\[0.1in]$ \sum_{n=1}^{\infty}  (-1)^n \frac{\cancelto{1}{3^n}}{n\cdot \cancelto{1}{3^n}}$
\\[0.1in]$ \sum_{n=1}^{\infty} \frac{(-1)^n}{n}$ is \boxed{$convergent by the Alternating Series Test$}, meaning that $-2$ is part of the interval of convergence
\\[0.1in]After testing endpoints, we can conclude the interval of convergence is  \boxed{-8  < x \leq -2}

\end{multicols*}
\section{$ f(x)=\dfrac{x^2}{3x+4}$}
\subsection*{Conversion to a power series}
To be easily converted to a power series $ \sum_{n=0}^{\infty}ar^n$, the equation will need the form of $ \frac{a}{1-r}$
\\[0.1in] We can first pull out an $ x^2$ from the fraction and get $ f(x) = x^2 \frac{1}{3x+4}$
\\[0.1in] Separating out a $ 4$ from the denominator get $ f(x) = x^2 \cdot \dfrac{1}{4} \cdot \dfrac{1}{\dfrac{3x}{4}+1}$ 
\\ Which can be rearranged as $ f(x) =  \cdot \dfrac{x^2}{4} \cdot \dfrac{1}{1+ \dfrac{3x}{4}}$
\\[0.05in] When $a=1$, $r =  - \dfrac{3x}{4}$, the power series can be represented as $ f(x) = \dfrac{x^2}{4}  \sum_{n=0}^{\infty}\left(-\dfrac{3x}{4}\right)^n$
\\[0.05in] $ f(x) = \dfrac{x^2}{4}  \sum_{n=0}^{\infty}\left(-\dfrac{3x}{4}\right)^n$
\\[0.05in] Distributing the power $ f(x) = \dfrac{x^2}{4}  \sum_{n=0}^{\infty}\dfrac{(-1)^n\cdot3^n\cdot x^n}{4^n}$
\\[0.05in] Since $x^2\cdot x^n = x^{n+2}$ and $4 \cdot 4^n = 4^{n+1}$, substituting them in we get $ f(x) =   \sum_{n=0}^{\infty}\dfrac{(-1)^n\cdot3^n\cdot x^{n+2}}{4^{n+1}}$
\\[0.05in] Similar to section 11.9 example 3, we can shift the indices $ f(x) =   \sum_{n=2}^{\infty}\dfrac{(-1)^{n-2}\cdot3^{n-2}\cdot x^{n}}{4^{n-1}}$
\\[0.05in] Plugging this back in, \boxed{\dfrac{x^2}{3x+4} =  \sum_{n=2}^{\infty}\dfrac{(-1)^{n-2}\cdot3^{n-2}\cdot x^{n}}{4^{n-1}}}
\subsection*{Finding radius of convergence}
Method: Ratio Test $R =  \lim_{n\to\infty}\left| \frac{a_{n+1}}{a_n} \right|$ where the series converges if $|R|<1$
\\[0.1in]and where $a_n$ denotes the $n_{th}$ term of the series and r is the radius of convergence
\\[0.1in] $a_n= \dfrac{(-1)^{n-2}\cdot3^{n-2}\cdot x^{n}}{4^{n-1}}$
\\[0.1in] $R=\lim_{n\to\infty} \left|   \dfrac{\dfrac{(-1)^{n-1}\cdot3^{n-1}\cdot x^{n+1}}{4^{n}}}{\dfrac{(-1)^{n-2}\cdot3^{n-2}\cdot x^{n}}{4^{n-1}}}    \right|$
\\[0.1in] $R=\lim_{n\to\infty} \left|   \dfrac{(-1)^{\cancelto{}{n-1}}\cdot3^{\cancelto{}{n-1}}\cdot x^{\cancelto{}{n+1}}}{4^{\cancelto{}{n}}}\cdot \dfrac{\cancelto{}{4^{n-1}}}{\cancelto{1}{(-1)^{n-2}\cdot3^{n-2}\cdot x^{n}}}    \right|$
\\[0.1in] $R=\lim_{n\to\infty} \left|   \dfrac{-3x}{4} \right|$
\\[0.1in] $R=\left|   \dfrac{-3x}{4} \right|$
\\[0.1in] $\left|   \dfrac{-3x}{4} \right| = 1$
\\[0.1in] $\left|   -3x \right| = 4$
\\[0.1in] $3x = 4$
\\[0.1in] \boxed{r= \frac{4}{3}}

\subsection*{Finding interval of convergence}
The series converges if $\left|   \dfrac{-3x}{4} \right| < 1$
\\[0.1in] $-1< \dfrac{3x}{4} < 1$
\\[0.1in] $-4< 3x < 4$
\\[0.1in] $-\frac{4}{3}< x < \frac{4}{3}$
\\[0.1in]\textbf{Testing the endpoints:}
\begin{multicols*}{2}
    Upper Endpoint: $x=\frac{4}{3}$
    \\[0.05in]$ \sum_{n=2}^{\infty}\dfrac{(-1)^{n-2}\cdot3^{n-2}\cdot \frac{4}{3}^{n}}{4^{n-1}}$
    \\[0.05in]$ =\sum_{n=2}^{\infty}\dfrac{(-3)^{n-2}\cdot \frac{4}{3}^{n}}{4^{n-1}}$
    \\[0.05in]$ =\sum_{n=2}^{\infty}\dfrac{(-3)^{n-2}\cdot \frac{4}{3}^{n-2} \cdot \frac{4}{3}^2}{4^{n-2}\cdot4}$
    \\[0.05in]$ =\sum_{n=2}^{\infty}\dfrac{(-4)^{n-2}\cdot \frac{4}{3}^2}{4^{n-2}\cdot4}$
    \\[0.05in]$ =\sum_{n=2}^{\infty}\dfrac{(-1)^{n-2}\cdot {16}}{4\cdot 9}$
    \\[0.05in]$ =\sum_{n=2}^{\infty}\dfrac{(-1)^{n-2}\cdot {4}}{ 9}$
    \\[0.05in]This series is divergent by the Ratio Test, where the ratio is 1, meaning the upper endpoint can not be included in the interval of convergence
    \\[0.5in]\boxed{\textbf{Radius of Convergence: $\frac{4}{3}$}}\\\boxed{\textbf{Interval of Convergence: $-\frac{4}{3}< x < \frac{4}{3}$}}
    \vfill\null
    \columnbreak

    Lower Endpoint: $x=-\frac{4}{3}$
    \\[0.05in]$ \sum_{n=2}^{\infty}\dfrac{(-1)^{n-2}\cdot3^{n-2}\cdot -\frac{4}{3}^{n}}{4^{n-1}}$
    \\[0.05in]$ =\sum_{n=2}^{\infty}\dfrac{(-3)^{n-2}\cdot -\frac{4}{3}^{n}}{4^{n-1}}$
    \\[0.05in]$ =\sum_{n=2}^{\infty}\dfrac{(-3)^{n-2}\cdot -\frac{4}{3}^{n-2} \cdot -\frac{4}{3}^2}{4^{n-2}\cdot4}$
    \\[0.05in]$ =\sum_{n=2}^{\infty}\dfrac{\cancelto{1}{4^{n-2}}\cdot -\frac{4}{3}^2}{\cancelto{1}{4^{n-2}}\cdot4}$
    \\[0.05in]$ =\sum_{n=2}^{\infty}\dfrac{\left( -\frac{4}{3}\right)^2}{4}$
    \\[0.05in]$ =\sum_{n=2}^{\infty}\dfrac{16}{4\cdot 9}$
    \\[0.05in]$ =\sum_{n=2}^{\infty}\dfrac{4}{9}$
    \\[0.05in]This series is divergent by the Ratio Test, where the ratio is 1, meaning the lower endpoint can not be included in the interval of convergence
    \pagebreak
\end{multicols*}
\section{$f(x) = 3^{-2x}$}
\begin{multicols*}{2}
    \subsection*{Finding the first 6 coefficients $\frac{f^{(n)}(0)}{n!}$}
    \textbf{0:}$f(x) =  3^{-2x}= 9^{-x} = \frac{1}{9^x}$ at $x=0$
    \\[0.05in]$\frac{1}{9^0}=1$ then dividing by 0! $\frac{1}{0!}=$\boxed{1}
    \\[0.05in]\textbf{1:}$f'(x) = \frac{d}{dx} 3^{-2x}=\frac{d}{dx} 3^{-2x} =\frac{d}{dx} \frac{1}{3^{2x}}$
    \\[0.1in]using the quotient rule $\frac{d}{dx} \frac{1}{3^{2x}} = \frac{0\cdot3^{2x}-3^{2x}(2\ln3)}{3^{4x}}$
    \\[0.1in]$= \frac{-3^{2x}(2\ln3)}{3^{4x}}$
    \\[0.1in]$= \frac{-2\ln3}{3^{2x}}$ where $x=0$
    \\[0.1in]$= \frac{-2\ln3}{3^{2*0}}=-2\ln3$
    \\[0in] dividing by $1!$ we get $\boxed{\frac{-2\ln3} {1!}}$
    \\[0.1in]\textbf{2:}$f''(x) = \frac{d}{dx} \frac{-2\ln3}{3^{2x}}$
    \\[0.1in]Using the quotient rule, again \\[0.02in]$\frac{d}{dx} \frac{-2\ln3}{3^{2x}}= \frac{0\cdot 3^{2x} -(-2\ln3\cdot3^{2x}(2\ln3))}{3^{4x}}$
    \\[0.1in] $= \frac{2\ln3\cdot3^{2x}(2\ln3)}{3^{4x}}$
    \\[0.1in] $= \frac{(2\ln3)(2\ln3)}{3^{2x}}$ where $x=0$
    \\[0.1in] $= (2\ln3)(2\ln3) $ or $(2\ln(3))^2$
    \\[0in]Dividing by 2! we get $\boxed{\frac{(2\ln(3))^2}{2!}}$
    \\[0.1in]\textbf{3:} $f'''(x)= \frac{d}{dx} \frac{(2\ln(3))^2}{3^{2x}}$ 
    \\[0.1in]Using the quotient rule, again \\[0.05in]$\frac{d}{dx} \frac{(2\ln(3))^2}{3^{2x}} = \frac{0*3^{2x}-(2\ln(3))^2\cdot3^{2x}2\ln3}{3^{4x}}$ 
    \\[0.1in] $ = \frac{-(2\ln(3))^2\cdot3^{2x}2\ln3}{3^{4x}}$ 
    \\[0.1in] $ = \frac{-(2\ln(3))^3}{3^{2x}}$ where $x=0$
    \\[0.1in] = ${-(2\ln(3))^3}$
    \\[0in] dividing by $3!$ we get $\boxed{\frac{-(2\ln(3))^3} {3!}}$
    \vfill\null
    \columnbreak
    \textbf{4: }$f^{(4)}(x)= \frac{d}{dx} \frac{-(2\ln(3))^3}{3^{2x}}$
    \\[0.1in]Using the quotient rule, again $\frac{d}{dx} \frac{-(2\ln(3))^3}{3^{2x}} = \frac{0*3^{2x}-(-(2\ln(3))^3\cdot3^{2x}2\ln3}{3^{4x}}$ 
    \\[0.1in]$= \frac{(2\ln(3))^3\cdot3^{2x}2\ln3}{3^{4x}}$ 
    \\[0.1in]$= \frac{(2\ln(3))^4}{3^{2x}}$ where $x=0$, evaluates to ${(2\ln(3))^4}$
    \\\\[0in]Dividing by 4! we get $\boxed{\frac{(2\ln(3))^4}{4!}}$
    \\[0.1in]\textbf{5: }$f^{(5)}(x)= \frac{d}{dx} \frac{(2\ln(3))^4}{3^{2x}}$
    \\[0.1in]Using the quotient rule, for the last time \\[0.1in]$\frac{d}{dx}\frac{(2\ln(3))^4}{3^{2x}} = \frac{0*3^{2x}-(2\ln(3))^4\cdot3^{2x}2\ln3}{3^{4x}}$ 
    \\[0.1in]$ = \frac{0*3^{2x}-(2\ln(3))^4\cdot3^{2x}2\ln3}{3^{4x}}$ 
    \\[0.1in]$ = \frac{-(2\ln(3))^4\cdot3^{2x}2\ln3}{3^{4x}}$ 
    \\[0.1in]$ = \frac{-(2\ln(3))^5\cdot3^{2x}}{3^{4x}}$ 
    \\[0.1in]$ = \frac{-(2\ln(3))^5}{3^{2x}}$ where $x=0$, evaluates to ${-(2\ln(3))^5}$
    \\\\[0in]Dividing by 5! we get $\boxed{\frac{-(2\ln(3))^5}{5!}}$
    \\[0.05in]
    \vfill\null
    \pagebreak
    \textbf{First 6 coefficients:}
    \begin{enumerate}
        \setcounter{enumi}{-1}
        \item $1$
        \item $\frac{-2\ln3}{1!}$
        \item $\frac{(2\ln(3))^2}{2!}$
        \item $\frac{-(2\ln(3))^3}{3!}$
        \item $\frac{(2\ln(3))^4}{4!}$
        \item $\frac{-(2\ln(3))^5}{5!}$
    \end{enumerate}
    The emergent pattern seems to be \boxed{\frac{f^{(n)}(0)}{n!}=(-1)^n\frac{(2\ln(3))^n}{n!}}
    \subsection*{\textbf{The Maclaurin Series}}
    Using the known coefficient pattern, we can multiply the general term by $x^n$ and make it a summation \boxed{\sum_{n=0}^{\infty}\frac{f^{(n)}(0)}{n!}x^n=\sum_{n=0}^{\infty}(-1)^n\frac{(2\ln(3))^n}{n!}x^n}

\end{multicols*}
\section{Line L passes through $A(4, -3, 5)$ and $B(-2, -1, 8)$}
\subsection*{Vector Equation}
For the purposes of this problem, I will use the vector going in the direction from point $A$ to point $B$
\\We can create a vector $\vec{v}$ that is parallel to $\overrightarrow{AB}$
\\The variable $t$ will represent a scalar parameter to any equation in the scope of this problem that takes in a parameter as an input
\\This vector $\vec{v}$ can be found using the formula $B-A$ and is calculated using $\Vec{v} =\langle B_x-A_x, B_y-A_y, B_z-A_z\rangle$
\\$\vec{v}=\langle-2-4, -1-(-3), 8-5\rangle = \langle-6, 2, 3\rangle$
\\Now using $\vec{r} = \vec{r_0}+t\vec{v}$, where $r_0$ is the tail point of the vector, denoted by point $A(4, -3, 5)$ for this example, we can compute the tip of the vector given a parameter and the following vector equation:
\\[0.1in]$\vec{r} = (4, -3, 5) + t\langle-6,2,3\rangle$
\\[0.1in]$\vec{r} = (4, -3, 5) + \langle-6t,2t,3t\rangle$
\\[0.1in]\boxed{\vec{r} = \langle4-6t,2t-3,3t+5\rangle}
\subsection*{Parametric Equations}
Using Theorem \#2 in section 12.5 of the textbook, Parametric equations for a line through the point $(x_0, y_0, z_0)$ and parallel to the direction vector $\langle a,b,c\rangle$ are:
\\[0.05in]$x=x_0+at\qquad y=y_0+bt \qquad z=z_0+ct$
\\[0.05in]$(x_0,y_0,z_0)$ will be the tail of the vector, point $A(4, -3, 5)$, and $\langle a,b,c\rangle$  will be the respective components of defined vector $\vec{v}$, \\$\vec{v} = \langle-6, 2, 3\rangle$
\\[0.05in]By Theorem \#2 in section 12.5, \boxed{x=4-6t\qquad y=-3+2t\qquad z=5+3t}
\subsection*{Symmetric Equation}
To find a symmetric equation, we can solve all 3 of our parametric equations for the variable $t$, assuming none of the components of $\langle a,b,c\rangle$ are 0, which would introduce division by zero
\\$x=4-6t$\hfill$y=-3+2t$\hfill$z=5+3t$\hfill\null
\\[0.05in]$x-4=-6t$\hfill$y+3=2t$\hfill$z-5=3t$\hfill\null
\\[0.05in]\boxed{\frac{x-4}{-6}=t}\hfill\boxed{\frac{y+3}{2} =t}\hfill\boxed{\frac{z-5}{3}=t}\hfill\null
\\[0.05in] Combining these into a single equation to eliminate the parameter $t$, we get the Symmetric Equation \\[0.05in]\begin{center}\boxed{\frac{x-4}{-6}=\frac{y+3}{2}=\frac{z-5}{3}}\end{center}
\subsection{Intersecting the planes}
The line $\frac{x-4}{-6}=\frac{y+3}{2}=\frac{z-5}{3}$ intersects the $xy$ plane when $z=0$
\\[0.1in]$\frac{x-4}{-6}=\frac{y+3}{2}=\frac{-5}{3}$
\\[0.1in]$\frac{x-4}{-6}=\frac{-5}{3}$
\\[0.1in]${x-4}=\frac{-5\cdot -6}{3}$
\\[0.1in]${x-4}=10$
\\[0.1in]\boxed{x=14}
\\[0.1in]$\frac{y+3}{2}=\frac{-5}{3}$
\\[0.1in]${y+3}=\frac{-5\cdot 2}{3}$
\\[0.1in]${y+3}=\frac{-10}{3}$
\\[0.1in]${y}=\frac{-10}{3}-3$
\\[0.1in]\boxed{y=\frac{-19}{3}}
\\Making the $xy$ plane intersection coordinates \boxed{(14,\frac{-19}{3},0)}
\\The line $\frac{x-4}{-6}=\frac{y+3}{2}=\frac{z-5}{3}$ intersects the $xz$ plane when $y=0$
\\[0.1in]$\frac{x-4}{-6}=\frac{3}{2}=\frac{z-5}{3}$ 
\\[0.1in]$\frac{x-4}{-6}=\frac{3}{2}$ 
\\[0.1in]${x-4}=\frac{-6\cdot3}{2}$ 
\\[0.1in]${x-4}=\frac{-18}{2}$ 
\\[0.1in]${x-4}={-9}$ 
\\[0.1in]\boxed{{x}={-5}}
\\[0.1in]$\frac{3}{2}=\frac{z-5}{3}$ 
\\[0.1in]$\frac{3\cdot 3}{2}={z-5}$ 
\\[0.1in]$\frac{9}{2}={z-5}$ 
\\[0.1in]\boxed{z=\frac{19}{2}}
\\[0.1in]Making the intersection point to the $xz$ plane \boxed{(-5,0,\frac{19}{2})}
\\The line $\frac{x-4}{-6}=\frac{y+3}{2}=\frac{z-5}{3}$ intersects the $yz$ plane when $x=0$
\\[0.1in]$\frac{-4}{-6}=\frac{y+3}{2}=\frac{z-5}{3}$
\\[0.1in]$\frac{-4}{-6}=\frac{y+3}{2}=\frac{z-5}{3}$
\\[0.1in]$\frac{4}{6}=\frac{y+3}{2}=\frac{z-5}{3}$
\\[0.1in]$\frac{4}{6}=\frac{y+3}{2}$
\\[0.1in]$\frac{8}{6}={y+3}$
\\[0.1in]$\frac{8}{6}-3={y}$
\\[0.1in]$y=\frac{-10}{6}$
\\[0.1in]\boxed{y=\frac{-5}{3}}
\\[0.1in]$\frac{-4}{-6}=\frac{z-5}{3}$
\\[0.1in]$\frac{-4\cdot 3}{-6}={z-5}$
\\[0.1in]$\frac{-12}{-6}={z-5}$
\\[0.1in]$2={z-5}$
\\[0.1in]\boxed{z=7}
\\[0.1in]Making the intersection point to the $yz$ plane \boxed{(0,\frac{-5}{3}, 7)}
\subsection{Distance to Point $C(-1,4,-6)$}
This can be calculated using $d=\frac{|\vec{v}\times\vec{BC}|}{|\vec{v}|} $ where 
\\[0.1in]$\vec{v}= \langle-6, 2, 3\rangle$ and $\vec{BC}=C-B =\langle-1-(-2), 4-(-1), -6-8\rangle = \langle1, 5, -14\rangle$
\textbf{$\vec{v}\times\vec{BC}$}
\\[0.1in]Starting off with the matrix
$\vec{v} \times \vec{BC} = $
$\begin{vmatrix}
i & j & k \\
-6 & 2 & 3 \\
1 & 5 & -14
\end{vmatrix}$
\\[0.05in]$\vec{v}\times\vec{BC} = i(2\cdot-14-3\cdot5)-j(-6\cdot-14-3\cdot 1)+k(-6\cdot5-2\cdot1)$
\\[0.05in]$\vec{v}\times\vec{BC} = i(-28-15)-j(84-3)+k(-30-2)$
\\[0.05in]$\vec{v}\times\vec{BC} = i(-43)-j(81)+k(-32)$
\\[0.05in]$\vec{v}\times\vec{BC} = \langle-43,-81,-32\rangle$
\\Finding the magnitude is easy using the Pythagorean theorem for 3 dimensions
\\$|\vec{v}\times\vec{BC}|=\sqrt{(-43)^2+(-81)^2+(-32)^2}$
\\$|\vec{v}\times\vec{BC}|=\sqrt{1849+6561+1024}$
\\$|\vec{v}\times\vec{BC}|=\boxed{\sqrt{9434}}$
\\[0.05in]Now to divide it by the magnitude of $\vec{v}$
\\[0.05in]$|\vec{v}| = \sqrt{(-6^2)+2^2+3^2}$
\\$|\vec{v}| = \sqrt{36+4+9}$
\\$|\vec{v}| = \sqrt{49}$
\\$|\vec{v}| = 7$
\\\boxed{d=\frac{\sqrt{9434}}{7}} units


\pagebreak
\section{Maclaurin Series for $\tan^{-1}x=\sum_{n=0}^{\infty}\frac{(-1)^n\cdot x^{2n+1}}{2n+1}$}
\subsection*{Power Series for $f(x)=x^3\cdot\tan^{-1}(x^2)$}
Given that we know the Maclaurin Series for $\tan^{-1}(x)$, modifying the power series for our new equation is straightforward
\\\textbf{Modifying the x}
As $x$ is not the index variable in the infinite series, it is fully permeable through the membrane of the sigma function. This will be accomplished by directly substituting the variable, similar to what was shown in example 11 in Section 11.10 in the textbook
\\$\tan^{-1}(x)=\sum_{n=0}^{\infty}\frac{(-1)^n\cdot x^{2n+1}}{2n+1}$
\\$x^3 \cdot \tan^{-1}(x^2)=x^3\sum_{n=0}^{\infty}\frac{(-1)^n\cdot (x^2)^{2n+1}}{2n+1}$
\\Multiplying together the stacked exponents: $x^3\cdot \tan^{-1}(x^2)=x^3\sum_{n=0}^{\infty}\frac{(-1)^n\cdot x^{4n+2}}{2n+1}\qquad$ as $2\cdot 2n+1=4n+2$
\\[0.1in]Moving the $x^3$ across the sigma function membrane, we get $x^3\cdot \tan^{-1}(x^2)=\sum_{n=0}^{\infty}\frac{(-1)^n\cdot x^{4n+2}\cdot x^3}{2n+1}\qquad$
\\[0.1in]Simplifies to $x^3\cdot \tan^{-1}(x^2)=\sum_{n=0}^{\infty}\frac{(-1)^n\cdot x^{4n+2+3}}{2n+1}\qquad$
\\[0.1in]$x^3\cdot \tan^{-1}(x^2)=\sum_{n=0}^{\infty}\frac{(-1)^n\cdot x^{4n+2+3}}{2n+1}\qquad$
\\[0.1in]\boxed{x^3\cdot \tan^{-1}(x^2)=\sum_{n=0}^{\infty}\frac{(-1)^n\cdot x^{4n+5}}{2n+1}\qquad}
\\[0.1in]\textbf{Checking for Convergence in the interval of integration} 
\\[0.05in]Using the Ratio test
\\[0.05in]Using $a_n = \frac{(-1)^n\cdot x^{4n+5}}{2n+1}$ we can do the ratio test $R =\lim_{n\to\infty} \left|\frac{a_{n+1}}{a_n}\right|$ and the series converges for $R < 1$
\\[0.05in]$R=\lim_{n\to\infty} \left|\frac{\frac{(-1)^{n+1}\cdot x^{4{(n+1)}+5}}{2{(n+1)}+1}}{\frac{(-1)^n\cdot x^{4n+5}}{2n+1}}\right|$
\\[0.05in]$R=\lim_{n\to\infty} \left|\frac{(-1)^{\cancelto{}{n+1}}\cdot x^{\cancelto{4}{4n+9}}}{2{(n+1)}+1}\cdot\frac{2n+1}{\cancelto{1}{(-1)^n}\cdot \cancelto{1}{x^{4n+5}}}\right|$
\\[0.05in]$R=\lim_{n\to\infty} \left|\frac{-x\cdot(2n+1)}{2n+3}\right|$
\\[0.05in]$R=|-x| \cancelto{1}{\lim_{n\to\infty} \left|\frac{(2n+1)}{2n+3}\right|}$
\\$R=|-x|$
\\$-1<-x<1$
\\$1>x>-1$ 
\\Since the series is for sure convergent for all values between, but not including  $x=-1$ and $x=1$, we know it is convergent for the interval $[0,\frac{1}{2}]$
\\[1in]
\textbf{Integrating $\int_{0}^{\frac{1}{2}}x^3\cdot \tan^{-1}(x^2)\;dx$}
\\Given the power series for $x^3\cdot \tan^{-1}(x^2)$ as $\sum_{n=0}^{\infty}\frac{(-1)^n\cdot x^{4n+5}}{2n+1}$
\\[0.1in]We can start off with $\int_{0}^{\frac{1}{2}}x^3\cdot \tan^{-1}(x^2)\;dx = \int_{0}^{\frac{1}{2}}\sum_{n=0}^{\infty}\frac{(-1)^n\cdot {x^{4n+5}}}{2n+1}dx$
\\[0.1in]We can move everything not bound to the variable $x$ out of the integrand, including the sigma function
\\[0.1in]$= \sum_{n=0}^{\infty}\frac{(-1)^n}{2n+1}\int_{0}^{\frac{1}{2}} x^{4n+5}dx$
\\[0.1in]\textbf{Evaluating the integral using the power rule}
\\[0.05in]$= \sum_{n=0}^{\infty}\frac{(-1)^n}{2n+1}\int_{0}^{\frac{1}{2}} x^{4n+5}dx$
\\[0.05in]$=\sum_{n=0}^{\infty}\frac{(-1)^n}{2n+1}\frac{x^{4n+6}}{4n+6} \bigg|_{x=0}^{x=\frac{1}{2}}$
\\[0.05in]$=\sum_{n=0}^{\infty}\frac{(-1)^n}{2n+1}\frac{(\frac{1}{2})^{4n+6}}{4n+6} -\cancelto{0}{\sum_{n=0}^{\infty}\frac{(-1)^n}{2n+1}\frac{0^{4n+6}}{4n+6}} $
\\[0.05in]=\boxed{\sum_{n=0}^{\infty}\frac{(-1)^n}{2n+1}\frac{(\frac{1}{2})^{4n+6}}{4n+6}}
\\[0.1in]Using a calculator, we can calculate the values of each term to find where $|$Error$| < 1.0\cdot 10^{-9} $or$ \frac{1}{1,000,000,000}$
\begin{enumerate}
    \setcounter{enumi}{-1}
    \item $\frac{1}{384} = 0.0026041666$
    \item $\frac{-1}{30720}=-0.000032552083$
    \item $\frac{1}{1146880}=8.7193080357\cdot 10^{-7}$
    \item $\frac{-1}{33030144} =-3.0275375124\cdot 10^{-8}$
    \item $\frac{1}{830472192} = 1.204134237886 \cdot 10^{-9}$
    \item $\frac{-1}{19193135104} = -5.210196221625\cdot 10^{-11}$ - This term is the error as it is the first term below $10^{-9}$
\end{enumerate}
Giving us the series: \boxed{\frac{1}{384}-\frac{1}{30720}+\frac{1}{1146880}-\frac{1}{33030144}+\frac{1}{830472192}}
Or: \boxed{\sum_{n=0}^{4}\frac{(-1)^n}{2n+1}\frac{(\frac{1}{2})^{4n+6}}{4n+6}}
\\[0.1in] The output from this summation is \boxed{0.00257246}
\vfill\null\pagebreak
\section{$A(2, 6, 5), B(-1, 3, 4), C(-1, -5, 7)$}
\subsection{Side Lengths}
\textbf{$\vec{|AB|}$}
\\[0.1in] $\vec{AB} = B-A$
\\[0.1in] $=\langle-1-2,3-6,4-5\rangle$
\\[0.1in] $=\langle-3,-3,-1\rangle$
\\[0.1in] Now for magnitude $=\sqrt{(-3)^2+(-3)^2+(-1)^2} =$ \boxed{\sqrt{19}}
\\[0.1in]\textbf{$\vec{|AC|}$}
\\[0.1in] $\vec{AC} = C-A$
\\[0.1in] $=\langle-1-2,-5-6,7-5\rangle$
\\[0.1in] $=\langle-3,-11,2\rangle$
\\[0.1in] Now for magnitude $=\sqrt{(-3)^2+(-11)^2+(2)^2} =\sqrt{9+121+4}$ \boxed{\sqrt{134}}
\\[0.1in]\textbf{$\vec{|BC|}$}
\\[0.1in] $\vec{BC} = C-B$
\\[0.1in] $=\langle-1-(-1),-5-3,7-4\rangle$
\\[0.1in] $=\langle0,-8,3\rangle$
\\[0.1in] Now for magnitude $=\sqrt{(0)^2+(-8)^2+(3)^2} =\sqrt{0+64+9}$ \boxed{\sqrt{73}}
\subsection{Angles}
Theorem: $|A\times B|=|A||B|\sin(\theta)$
\\[0.1in]$|\vec{AB}\times \vec{BC}|=|\vec{AB}||\vec{BC}|\sin(\theta)$
\\[0.1in]$\vec{AB} \times \vec{BC} = $
$\begin{vmatrix}
i & j & k \\
-3 & -3 & -1 \\
0 & -8 & 3
\end{vmatrix}$
\\$\vec{AB} \times \vec{BC} = i(-3\cdot3-(-1)\cdot -8)-j(-3\cdot 3-(-1)\cdot 0)+k(-3\cdot-8-3\cdot 0)$
\\$\vec{AB} \times \vec{BC} = i(-17)-j(-9)+k(24)$
\\$\vec{AB} \times \vec{BC} = \langle-17,9,24\rangle$
\\$|\vec{AB} \times \vec{BC}| = \sqrt{(-17)^2+9^2+(24)^2}$
\\$|\vec{AB} \times \vec{BC}| = \sqrt{(-17)^2+9^2+(24)^2}$
\\[0.1in]$|\vec{AB} \times \vec{BC}| = \sqrt{946}$
\\[0.1in]$\sqrt{946}=\sqrt{19}\sqrt{73}\sin\theta$
\\[0.1in]$\frac{\sqrt{946}}{\sqrt{19}\sqrt{73}}=\sin\theta$
\\[0.1in]$\arcsin\frac{\sqrt{946}}{\sqrt{19}\sqrt{73}}=\theta$
\\[0.1in]$55.68\degree=\theta$ 
\\[0.1in]However, since that input exceeds the maximum possible angle that arcsin can provide ($90\degree$), we can to subtract the output from $180\degree$ to get the final angle. This is true since $sin(x)$ is horizontally symmetric around the points $x=\frac{\pi}{2}+n\pi$
\\[0.1in]$180\degree-55.68\degree=$ \boxed{\theta =124.32\degree} for angle $\angle ABC$
\\[0.1in]$|\vec{AC}\times \vec{BC}|=|\vec{AC}||\vec{BC}|\sin(\theta)$
\\[0.1in]$\vec{AC} \times \vec{BC} = $
$\begin{vmatrix}
i & j & k \\
-3 & -11 & 2 \\
0 & -8 & 3
\end{vmatrix}$
\\[0.1in]$\vec{AC} \times \vec{BC} = i(-11\cdot3-2\cdot -8)-j(-3\cdot 3-2\cdot 0)+k(-3\cdot-8-11\cdot 0)$
\\[0.1in]$\vec{AC} \times \vec{BC} = i(-17)-j(-9)+k(24)$
\\[0.1in]$\vec{AC} \times \vec{BC} = \langle-17,9,24\rangle$
\\[0.1in]$|\vec{AC} \times \vec{BC}| = \sqrt{(-17)^2+(9)^2+(24)^2}$
\\[0.1in]$|\vec{AC} \times \vec{BC}| = \sqrt{946}$
\\[0.1in]$\sqrt{946} = \sqrt{134}\sqrt{73}\sin\theta$
\\[0.1in]$\frac{\sqrt{946}}{\sqrt{134}\sqrt{73}} = \sin\theta$
\\[0.1in]$\arcsin\frac{\sqrt{946}}{\sqrt{134}\sqrt{73}} = \theta$
\\[0.1in]\boxed{\theta = 18.12\degree} for angle $\angle ACB$
\\[0.1in]$|\vec{AB}\times \vec{AC}|=|\vec{AB}||\vec{AC}|\sin(\theta)$
\\[0.1in]$\vec{AB} \times \vec{AC} = $
$\begin{vmatrix}
i & j & k \\
-3 & -3 & -1 \\
-3 & -11 & 2
\end{vmatrix}$
\\[0.1in]$\vec{AB} \times \vec{AC} = i(-3\cdot2-(-1)\cdot -11)-j(-3\cdot 2-(-1)\cdot -3)+k(-3\cdot-11-(-3)\cdot -3)$
\\[0.1in]$\vec{AB} \times \vec{AC} = i(-6-11)-j(-6-3)+k(33-9)$
\\[0.1in]$\vec{AB} \times \vec{AC} = i(-17)-j(-9)+k(24)$
\\[0.1in]$\vec{AB} \times \vec{AC} = \langle-17,99,24\rangle$
\\[0.1in]$|\vec{AB} \times \vec{AC}| = \sqrt{(-17)^2+(9)^2+24^2}$
\\[0.1in]$|\vec{AB} \times \vec{AC}| = \sqrt{946}$
\\[0.1in]$\sqrt{946}=\sqrt{19}\sqrt{134}\sin\theta$
\\$\frac{\sqrt{946}}{\sqrt{19}\sqrt{134}}=\sin\theta$
\\$\arcsin\frac{\sqrt{946}}{\sqrt{19}\sqrt{134}}=\theta$
\\[0.1in]\boxed{\theta=37.56\degree}
\subsection{Linear Equation of Plane Containing points A,B,C}
Using equation $a(x-x_0)+b(y-y_0)+c(z-z_0)=0$ where $a, b, c$ are components of the normal vector $\vec{n}$ and $(x_0,y_0,z_0)$ are coordinates for a reference points, in this case point $A(2, 6, 5)$
\\[0.1in]$\vec{AB} \times \vec{AC} = $
$\begin{vmatrix}
i & j & k \\
-3 & -3 & -1 \\
-3 & -11 & 2
\end{vmatrix}$
\\[0.1in]$\vec{AB} \times \vec{AC} = i(-3\cdot2-(-1)\cdot -11)-j(-3\cdot 2-(-1)\cdot -3)+k(-3\cdot-11-(-3)\cdot -3)$
\\[0.1in]$\vec{AB} \times \vec{AC} = i(-6-11)-j(-6-3)+k(33-9)$
\\[0.1in]$\vec{AB} \times \vec{AC} = i(-17)-j(-9)+k(24)$
\\[0.1in]$\vec{AB} \times \vec{AC} = \langle-17,-9,24\rangle$
\\[0.1in] Let $\vec{n} = \langle a,b,c\rangle=\langle-17,-9,24\rangle$
\\[0.1in]Plugging things in, we get ${-17(x-2)+9(y-6)+24(z-5)=0}$
\\[0.1in]${-17x+34+9y-54+24z-120)=0}$
\\[0.1in]\boxed{-17x+9y+24z=140}
\subsection{Coplanarity with $D(-2, 6,-4)$}
To see if point $D(-2, 6,-4)$ lies on the plane $-17(x-2)+9(y-6)+24(z-5)=0$, we can plug in the coordinates
\\[0.1in]$-17(-2-2)+9(6-6)+24(-4-5)=0$
\\[0.1in]$-17(-4)+9(0)+24(-9)=0$
\\[0.1in]$68-216=0$
\\[0.1in]$-148=0$ This is obviously not true, meaning point \boxed{D $ is not coplanar with points $ A,B,C}
\subsection{Volume of the Parallelepiped}
Volume can be found using the formula provided by the textbook $V=|\vec{a}\cdot (\vec{b}\times \vec{c})|$
\\Lets refresh on our points and vectors: 
\\Points: $A(2, 6, 5), B(-1, 3, 4), C(-1, -5, 7), D(-2,6,-4)$
\\Vectors:$\vec{AB}=\langle-3,-3,-1\rangle$, $\vec{AC}=\langle-3,-11,2\rangle$, $\vec{BC}=\langle0,-8,3\rangle$
\\Calculating vector $\vec{AD}$
\\$\vec{AD} = D-A$
\\$\vec{AD} = \langle-2-2,6-6, -4-5\rangle$
\\$\vec{AD} = \langle-4,0, -9\rangle$
\\Now we can use the Volume Formula using $\vec{a} = \vec{AB}, \vec{b} = \vec{AC}, \vec{c} = \vec{AD}$
\\\textbf{Calculating $\vec{b}\times\vec{c}$}
\\[0.1in]$\vec{b}\times\vec{c} = \vec{AC}\times\vec{AD}$
\\[0.1in]$\vec{AC} \times \vec{AD} = $
$\begin{vmatrix}
i & j & k \\
-3 & -11 & 2\\
-4 & 0 & -9
\end{vmatrix}$
\\[0.1in]$\vec{AC} \times \vec{AD} = i(-11\cdot-9-2\cdot 0)-j(-3\cdot -9-(-4)\cdot 2)+k(-3\cdot0-(-11)\cdot -4)$
\\[0.1in]$\vec{AC} \times \vec{AD} = i(99)-j(35)+k(-44)$
\\[0.1in]$\vec{AC} \times \vec{AD} = \langle99,-35,-44\rangle$
\\[0.1in]$\vec{b} \times \vec{c} = \langle99,-35,-44\rangle$
\\[0.1in]\textbf{Calculating the dot product with $\vec{a} \cdot \vec{b}\times\vec{c}$}
\\[0.1in]$\langle-3,-3,-1\rangle\cdot\langle99,-35,-44\rangle$
\\[0.1in]$=-3\times99+-3\times-35+-1\times-44$
\\[0.1in]$=-297+105+44$
\\[0.1in]$=-148$
\\[0.1in]To top it all off, take the absolute value of the result $V= |-148| =$ \boxed{148 $ units$^3}


\end{document}
